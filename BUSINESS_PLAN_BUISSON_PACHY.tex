\documentclass[10pt]{article}
\usepackage[utf8]{inputenc}
\usepackage[francais]{babel}
\usepackage[T1]{fontenc}
\usepackage{graphicx}
\usepackage{lmodern}
\usepackage{url}
\usepackage{underscore}
\usepackage{array}
\usepackage{float}
\usepackage{listings}
\usepackage{regexpatch}
\usepackage[scale=0.75]{geometry}
\usepackage{color}
\usepackage{bigcenter}

\graphicspath{{../images/}}

\title{\textbf{Business Plan}\\{ {\small {\bsc {- Res Publica -}}}}}
\author{Fabien \bsc{Buisson} - Ravi \bsc{Pachy}}
\date{Février 2014}

\begin{document}

\renewcommand{\contentsname}{Sommaire}
\maketitle
\thispagestyle{empty}

\newpage

\tableofcontents

\newpage
\pagestyle{headings}

\section{Présentation}
\label{sec:presentation}
En dernier

% subsection section name (end)

% section Contexte, historique (end)

\section{Équipe \& fonctions}
\label{sec:equipe_fonctions}

\subsection{Introduction}
\label{sub:equipe_intro}
y'aura pas besoin de monde
vla les fonctions
% subsection Introduction (end)

\subsection{Webmaster}
\label{sec:webmaster}
Sera en charge de la réalisation et de la maintenance du site web et permettra le développement d'applications servant à l'entreprise.

% subsection Webmaster (end)

\subsection{Chargé de communication}
\label{sec:communication}
Sa mission sera d'obtenir des contrats directement avec l'habitant ou l'organisme. Il devra avoir préalablement étudier la région afin de proposer et répondre au mieux aux attentes de la clientèle, ainsi qu'être au courant des dernières méthodes et materiaux utilisés.

% subsection Chargé de communication (end)

\subsection{Veille}
\label{sec:}
\subsubsection{Technologique}
\label{ssub:}
L'objet d'une veille technologique sera primordia afin de pouvoir entrer en contact avec les meilleurs fournisseurs dans les différentes région d'action. Une étude régulière dans les différents secteurs (panneaux solaire, eoliennes...) devra être rigoureuse, le but étant de proposer le meilleur du marché, connaitre les avantages et inconvenients de chaque ressource est indispensable.

% subsubsection  (end)

\subsubsection{Fournisseurs}
\label{ssub:}
La recherche de fournisseurs en ressource se réalisera toujours en fonction de la région ciblée. Une étude sur les fournisseurs locaux, ou dans les régions alentours, se réalisera après avoir pris connaissance des meilleurs produits sur le marché actuel. Suite à un trés bon rapport qualité/prix, nous serons en mesure de proposer le meilleur du marché à des prix trés abordable.

% subsubsection  (end)
% subsection Veille (end)

\subsection{Resources humaines}
\label{sec:rh}
Le responsable des ressources humaines aura pour mission de recruter de la main d'oeuvre locale. Un contrat dans une région permet d'employer élécritien(s), plombier(s), maçon(s), agriculteur(s).., parmis les intérimaires locaux.
% subsection Resources humaines (end)

\subsection{Comptable}
\label{sec:}
Gestionnaire de toutes les dépenses et les recettes. Achat de fournitures de bureau, paiement des salaires, vente/achats de produits…, il contrôle tous les mouvements d’argent. En fin d’année, il élabore le bilan comptable de l’entreprise. 
% subsection Comptabilité (end)

\subsection{Social \& culturel}
\label{sec:}
Aide à la création des assos (statuts figés, supports, contacts, documents type)
Dialogue avec elles (gestion des sous communs)
Aide à la réalisation de projets pour tout le monde
% subsection Social \& culturel (end)

\subsection{Dirigeant}
\label{sec:dirigeant}
Ce rôle est strictement encadré par les limites et compétences qui lui sont attribuées par les instances décisionnaires de l'entreprise. À l'intérieur de ce cadre, le chef d'entreprise dispose d'une marge d'initiative pour diriger et conduire l'entreprise.
Il défend les intérêts des propriétaires de l'entreprise, mais aussi ceux des parties prenantes à l'entreprise (notamment les salariés et les prestataires de service oeuvrant pour elle)
Il défend aussi un collectif plus élargi pouvant comprendre les parties prenantes externes comme les fournisseurs et  les clients de l'entreprise.
% subsection Dirigeant (end)

% section Équipe & fonctions (end)

\section{Ressources matérielles}
\label{sec:ressources}
ce qu'on a besoin au départ (com, support)
création de la doc (gimp, latex)
hébergeur ...
% section Ressources (end)

\section{Étude du marché}
\label{sec:marche}

\subsection{Clients}
\label{sub:client}
on vise la campagne avec des jardins
% subsection Clients (end)

\subsection{Stratégie}
\label{sub:strategie}
on prospecte
on installe
on redistribue
% subsection Stratégie (end)

% section Marché (potentiel, stratégie, …) (end)

\section{Financier (futur et éventuellement passé)}
\label{sec:financier}
Voir les exemples du prof
% section Financier (futur et éventuellement passé) (end)

\section{Juridique}
\label{sec:juridique}
SARL/EURL, toussa
% section Juridique (end)

\section{Création de l'entreprise}
\label{sec:creation_entreprise}

\subsection{Locaux}
\label{sub:locaux}

% subsection Locaux (end)
\subsection{Assurance}
\label{sub:assurance}

% subsection assurance (end)

\subsection{Communication}
\label{sub:communication}

% subsection communication (end)

\subsection{Recrutement}
\label{sub:recrutement}

% subsection recrutement (end)

\subsection{Comptabilité \& Gestion}
\label{sub:comptabilité_gestion}

% subsection  comptabilité_gestion (end)

\subsection{Actions commerciales}
\label{sub:actions_commerciales}

% subsection actions_commerciales (end)

\subsubsection{Echéances fiscales}
\label{ssub:echéance_fiscales}
% subsubsection echéance_fiscales (end)

\subsubsection{Croissance}
\label{ssub:croissance}
% subsubsection croissance (end)

\subsubsection{Développement}
\label{ssub:developpement}
% subsubsection developpement (end)

\subsubsection{Difficultés}
\label{ssub:difficultés}
% subsubsection difficultés (end)






% section  Création de l'entreprise (end)

























\end{document}